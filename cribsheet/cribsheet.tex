\documentclass[a4paper]{report}

\begin{document}

\section{Randomisernade Algoritmer}

Something about random solutions.
\begin{itemize}
	\item Vad är en randomiserande algoritm?
	\item När använder man den?
	\item Ett exempel.
\end{itemize}

\subsection{Average Case Analysis}

\subsection{Algorithms That Behaves Randomly}

\subsection{Usefullness}
by allowing rondomized decision, underlying model more powerful.

Problems not solvable by deterministic algorithms may be solvable by randomized
algorithms.

Det som är nice med rand algo är att vi kan ta ett problem och hitta en lösning
väldigt fort men ändå ha en hum om hur bra resultatet förhåller sig till den
korrekta lösningen.

Rando algo kan vara användbara även om problemet inte är NP-hard, rand kan lösa
problem med mindre minne och ev snabbare än en deterministisk algoritm. I
distribuerade system måste enheter komunicera mycket men med rand kan man
förenkla det genom att minska mängden kommunikation och synkronisering. % TODO

\section{Approximerarnde Algoritmer}

Något om approximerande alogritmer.
\begin{itemize}
	\item Vad är en approximerarnde algoritm?
	\item När använder man den?
	\item Ett exempel.
\end{itemize}

\section{Komplexitet}
Hur man beräknar komplexitet.

\end{document}

