\documentclass[a4paper]{report}

\usepackage[utf8]{inputenc}
%\usepackage[T1]{fontenc}

\begin{document}

\section{Kapitel 13.1-13.2 - Randomized Algorithms}

Det finns två typer av randomiserade algoritmer:
\begin{itemize}
	\item Traditionella algoritmer som får slumpmässig input. Detta brukar kallas \emph{average-case analysis} eftersom man är intresserad av hur algoritmer presterar i medel istället för att som vanligt studera worst case input.
	\item Algoritmer som tar slumpmässiga beslut för att lösa problem med worst case input.
\end{itemize}

Det är denna andra typ av randomized algoritm vi kommer att diskutera.

\subsection{Användbarhet}
Algoritmer som tar slumpmässiga beslut kan vara användbara till en en mängd olika saker. Exempelvis:
\begin{itemize}
	\item Ta fram en slumpmässig lösning på ett NP-hardproblem på
	polynomisk tid genom att offra nogrannhet i svaret. Genom att studera
	väntevärdet för vår slumpmässigt valda lösning och sedan jämföra det
	med den korrekta lösningen kan vi få en uppfattning om hur svaret från
	vår algoritm förhåller sig till det korrekta svaret.

	\item Reglera hur ofta processer som använder en delad resurs ska
	försöka komma ut resursen utan att blockera någon annan.
\end{itemize}

\subsection{Contention resolution - Shared resource problem}

\subsubsection{Problem}
\begin{itemize}
	\item Vi har $n$ processer och en gemensam databas.
	\item Tiden är delad i rundor, diskret tid.
	\item Databasen tillåter endast en process i varje runda.
	\item Om två processer försöker komma åt databsen samtidigt låser de ute varandra.
	\item Alla processer vill komma åt databasen så ofta som möjligt.
	\item Processerna kan inte kommunicera med varandra.
\end{itemize}
       
Hur ska processerna bete sig för att öka chansen att de får använda databasen?

\subsubsection{Algoritm}
I början av varje runda försöker varje process med sannolikheten $p$ komma åt
databasen.

\subsubsection{Analys av algoritmen}
Vi vill ta reda på det optimala värdet på $p$. Det intuitiva fallet är $p =
\frac{1}{n}$.

Låt $A$ beskriva händelsen att en process försöker komma åt databasen. Sannolikheten för detta beskrivs då av $P(A) = p$. Sannolikheten för att en process \emph{inte} försöker komma åt databasen är således $P(\overline{A}) = 1 - p$. 

Men vad vi egentligen bryr oss om är ifall en process lyckas med att komma åt databasen i en runda. Låt $S$ beskriva denna händelse. Eftersom en process lyckas komma åt databasen är likvärdigt med att just den och ingen annan process försökt och dessa händelser alla är oberoende blir sannolikheten för $S$ således:
\begin{equation}
	P(S) = f(p) = p(1-p)^{n-1}
\end{equation}

Genom att derivera $f(p)$ och sätta till $0$ får man fram maximala värdet på
$p$ vilket är $\frac{1}{n}$.



Sannolikheten att en process lyckas
\begin{equation}
	f(\frac{1}{n}) = \frac{1}{n} \cdot (1 - \frac{1}{n})^{(n-1)} \geq
	\frac{1}{n} \cdot \frac{1}{e} \leq \frac{1}{2n}\mbox{, eftersom} (1 -
	\frac{1}{n})^{(n-1)} \geq \frac{1}{e}
\end{equation}

Sannolikheten att processen $i$ får köra i rundan $t$:
\begin{equation}
	Pr(S(i,t)) = \frac{1}{n}(1 - \frac{1}{n})^{(n-1)}
\end{equation}
Sannolikheten att processen, $i$, inte får köra under $t$ rundor
\begin{equation}
	Pr(F(i,t)) = (1 - Pr(S(i,t)))^t = (1 - \frac{1}{n}(1 -
	\frac{1}{n})^{n-1})^t \leq (1 - \frac{1}{en})^t
\end{equation}

Så om vi sätter $t = \lceil{}en\rceil$ så
\begin{equation}
	Pr(F(i,t)) \leq (1 - \frac{1}{en})^{\lceil{}en\rceil} \leq (1 - \frac{1}{en})^{en}
	\leq \frac{1}{e}
\end{equation}
Vilket betyder att sannolkheten att en process inte får tillgång till databasen
under de första $en$ rundorna är mindre än $\frac{1}{e}$ ($O(1)$).

Om vi sätter $t = \lceil{}en\rceil \cdot (c\cdot{}ln(n))$
\begin{equation}
	Pr(F(i,t)) \leq (1 - \frac{1}{en})^{\lceil{}en\rceil \cdot (c\cdot{}ln(n))}
	\leq ((1 - \frac{1}{en})^{en})^{c\cdot{}ln(n)} \leq
	\frac{1}{e^{c\cdot{}ln(n)}} = \frac{1}{n^c}
\end{equation}
Detta betyder att om tiden ökar $O(n\cdot{}log(n))$ är sannolikheten att en
process inte har fått tillgång till databasen mycket liten och minskar med $n$,
$O(\frac{1}{n})$.


\section{Approximerarnde Algoritmer}

Något om approximerande alogritmer.
\begin{itemize}
	\item Vad är en approximerarnde algoritm?
	\item När använder man den?
	\item Ett exempel.
\end{itemize}

\section{Komplexitet}
Hur man beräknar komplexitet.

\end{document}

