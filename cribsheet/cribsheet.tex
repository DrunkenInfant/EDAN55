\documentclass[a4paper]{report}

\usepackage[utf8]{inputenc}
%\usepackage[T1]{fontenc}

\begin{document}

\section{Randomisernade Algoritmer}

Something about random solutions.
\begin{itemize}
	\item Vad är en randomiserande algoritm?
	\item När använder man den?
	\item Ett exempel.
\end{itemize}

\subsection{Average Case Analysis}

\subsection{Algorithms That Behaves Randomly}

\subsection{Usefullness}
by allowing rondomized decision, underlying model more powerful.

Problems not solvable by deterministic algorithms may be solvable by randomized
algorithms.

Det som är nice med rand algo är att vi kan ta ett problem och hitta en lösning
väldigt fort men ändå ha en hum om hur bra resultatet förhåller sig till den
korrekta lösningen.

Rando algo kan vara användbara även om problemet inte är NP-hard, rand kan lösa
problem med mindre minne och ev snabbare än en deterministisk algoritm. I
distribuerade system måste enheter komunicera mycket men med rand kan man
förenkla det genom att minska mängden kommunikation och synkronisering. % TODO

\subsection{Case study - Shared resource}
\subsubsection{Problem}
\begin{itemize}
	\item Vi har $n$ processer och en gemensamm databas.
	\item Alla processer vill komma åt databasen så ofta som  möjligt.
	\item Databasen tillåter endast en process i taget.
	\item Tiden är delad i rundor, diskret tid.
	\item Processerna kan inte kommunicera med varandra.
	\item Om två processer försöker komma åt databsen sammtidigt kommer de låsa
		ute varandra.
\end{itemize}
       
Hur ska processerna bete sig för att öka chansen att de får använda databasen?
\subsubsection{Designing a solution}
I början av varje runda försöker varje process med sannolikheten $p$ komma åt
databasen. Detta är en enkel lösing och ska visa sig vara väldigt bra.

\subsubsection{Analys av algoritmen}
Vi vill ta reda på det optimala värdet på $p$. Det intuitiva fallet är $p = \frac{1}{n}$.

Sannolikheten att en process för köra är $p \cdot P(<att inga andra processer
forsoker anv.\ database>)$ eftersom dessa händelser är oberoende.
\begin{equation}
	\mbox{Sannolikheten att en process lyckas} = p \cdot (1 - p)^{n - 1} = f(p)
\end{equation}

Genom att derivera $f(p)$ och sätta till $0$ får man fram maximala värdret på
$p$ vilket är $\frac{1}{n}$.

Sannolikheten att en process lyckas
\begin{equation}
	f(\frac{1}{n}) = \frac{1}{n} \cdot (1 - \frac{1}{n})^{(n-1)} \geq
	\frac{1}{n} \cdot \frac{1}{e} \leq \frac{1}{2n}\mbox{, eftersom} (1 -
	\frac{1}{n})^{(n-1)} \geq \frac{1}{e}
\end{equation}

Sannolikheten att processen $i$ får köra i rundan $t$:
\begin{equation}
	Pr(S(i,t)) = \frac{1}{n}(1 - \frac{1}{n})^{(n-1)}
\end{equation}
Sannolikheten att processen, $i$, inte får köra under $t$ rundor
\begin{equation}
	Pr(F(i,t)) = (1 - Pr(S(i,t)))^t = (1 - \frac{1}{n}(1 -
	\frac{1}{n})^{n-1})^t \leq (1 - \frac{1}{en})^t
\end{equation}

Så om vi sätter $t = roof(en)$ så
\begin{equation}
	Pr(F(i,t)) \leq (1 - \frac{1}{en})^{roof(en)} \leq (1 - \frac{1}{en})^{en} \leq
	\frac{1}{e}
\end{equation}
Vilket betyder att sannolkheten att en process inte får tillgång till databasen
under de första $en$ rundorna är mindre än $\frac{1}{e}$.


\section{Approximerarnde Algoritmer}

Något om approximerande alogritmer.
\begin{itemize}
	\item Vad är en approximerarnde algoritm?
	\item När använder man den?
	\item Ett exempel.
\end{itemize}

\section{Komplexitet}
Hur man beräknar komplexitet.

\end{document}

